% % % % % % % % % % % % % % % % % % % % % % % % % % % % % % % % % % % % % % % % % % 
% Exemple de rapport de master
% Version originale :  A. Tarantola, Octobre 2004
% Version modifiée :   A. Fournier et G. Moguilny, Mars 2012... Novembre 2025
% % % % % % % % % % % % % % % % % % % % % % % % % % % % % % % % % % % % % % % % % % 

\documentclass{ipgpmaster}













% --- Fix keywords macros (if class expects them) ---
\providecommand{\keyslabel}{Palabras clave:}
\providecommand{\keywords}{}



% Pour saisir directement les lettres accentuées :
% \usepackage[applemac]{inputenc} % pour utilisateurs mac
% \usepackage[ansinew]{inputenc}  % pour Windows ANSI
% \usepackage[latin1]{inputenc}   % pour Unix Latin1
\usepackage[T1]{fontenc}
\usepackage{amsmath}
\usepackage{graphicx}
\usepackage{pdflscape} % rota la página en el PDF final
\usepackage{tabularx}  
\usepackage{array}
\usepackage{xcolor}
\definecolor{feriadosGuinda}{RGB}{128, 0, 32} % guinda
\definecolor{pucpBlue}{RGB}{35,71,160}
\definecolor{ecoAqua}{RGB}{58,170,150}

\usepackage{fancyhdr} % ya lo carga la clase, pero no pasa nada si está

\fancypagestyle{cronograma_sin_anio}{
  \fancyhf{} % limpia header y footer
  \renewcommand{\headrulewidth}{0pt}
  \renewcommand{\footrulewidth}{0pt}
}

\definecolor{feriadosRed}{RGB}{255,0,0}
% \bibliography{master_AT}  % pour bibtex
\addbibresource{masterBib.bib} % pour biblatex
\selectlanguage{spanish}

\addto\captionsspanish{%
  \renewcommand{\tablename}{Tabla}%
  \renewcommand{\figurename}{Figura}%
}

\begin{document}

\checkyears

\vspace*{5mm}

%\setkeys{Gin}{draft} % images non affichées
\setkeys{Gin}{draft=false} % images affichées

%\linenumbers % Numerotation des lignes pour relecture
%Select here the language 



\def\author{ÁREA DE EVENTOS Y RELACIONES INSTITUCIONALES}
%\footnote{
%Albert Tarantola (1949-2009) : Géophysicien français d'origine espagnole, dont les travaux %ont notamment porté sur la théorie du problème
%inverse. Il a effectué sa carrière à l'IPGP, étant notamment responsable du DEA (diplôme %d'études approfondies, l'ancêtre du master 2) de géophysique interne entre
%1991 et 1996. En tant que responsable du DEA, il a encouragé l'utilisation de \LaTeX\ pour %la rédaction des rapports de stage, et rédigé les recommandations reprises dans ce document.
%}}
\def\title{PLAN DE TRABAJO - 2026}
\def\shorttitle{Eventos y R.R.I.I.}
\def\unit{Económica}
\def\team{Eventos y Relaciones Institucionales}
\def\spe{Organización Estudiantil} % Parcours
\def\supervisor{Del Castillo, D., Yika, J., Guzmán, C.}
\def\mydate{Enero 2, 2026}


\Entete
\begin{abstract}

El presente documento detalla el Plan de trabajo de ... organizacion ... corresponde al periodo 2026-0
El objetivo central de esta gestión será la transformación estratégica del área en un núcleo dinámico, integrador e influyente, orientado a operar con eficacia y profesionalismo. Para ello, nos proponemos constituir un nexo central que fomente una comunidad académica cohesionada y comprometida, fortaleciendo tanto los vínculos internos como las alianzas institucionales externas.

A través de la ejecución de un programa estructurado de actividades, se buscará proyectar una imagen renovada de la organización, caracterizada por su relevancia académica, vitalidad institucional y rigor profesional. El plan se articula en torno a los siguientes ejes: la reconstrucción del capital social y la confianza e integración interna, la gestión estratégica de relaciones con miembros y socios institucionales, y la gestión proactiva de la imagen y reputación organizacional.

En consecuencia, este plan busca relegitimar el área como un pilar indispensable para la unidad interna y la proyección externa, reafirmando así su valor tangible y contribución estratégica a los objetivos generales de la organización.

\renewcommand{\keywords}{finanzas; riesgo; LyX; GitHub}
\end{abstract}
\vfill

\rule{\linewidth}{0.4pt}\\[3mm]
\textit{
contacto institucional:
a20200989@pucp.edu.pe \textsuperscript{*} /
 jyika@pucp.edu.pe \textsuperscript{†} /
 csguzman@pucp.edu.pe \textsuperscript{‡}
}


\newpage

\renewcommand{\contentsname}{Tabla de contenidos}

\tableofcontents
\newpage

---

\section{Justificación del Proyecto}
La realización de una foto institucional de directores responde a la necesidad de proyectar una imagen coherente, profesional y representativa del liderazgo de Económica. Contar con material visual institucional actualizado fortalece la identidad organizacional, mejora la percepción externa y refuerza el sentido de pertenencia interna.

Asimismo, este proyecto constituye una acción estratégica de alto impacto y bajo costo, alineada con los objetivos del Área de Eventos y Relaciones Institucionales en materia de imagen, comunicación y posicionamiento institucional.

---

\section{Objetivos del Proyecto}

\subsection{Objetivo General del proyecto}
Fortalecer la imagen institucional y el posicionamiento organizacional de Económica mediante la realización de una sesión fotográfica profesional de sus directores.

\subsection{Objetivos Específicos alineados al plan de trabajo}


---

\section{Enfoque de Gestión y Orientación a Resultados}
El proyecto se ejecutará bajo un enfoque de gestión orientado a resultados, priorizando la obtención de productos verificables (fotografías institucionales editadas) y resultados organizacionales (mejora de la imagen institucional y mayor visibilidad externa).

Las acciones estarán alineadas a objetivos claros, con responsables definidos, tiempos establecidos e indicadores de seguimiento.

---

\section{Partes Implicadas y Roles}
\begin{itemize}
    \item \textbf{Dirección de Eventos y RR.II:} Coordinación general del proyecto y supervisión.
    \item \textbf{Subdirección de Eventos y RR.II:} Apoyo logístico y coordinación operativa.
    \item \textbf{Coordinación de Eventos y RR.II:} Seguimiento del cronograma y comunicación con los participantes.
    \item \textbf{Directores de Económica:} Participación en la sesión fotográfica.
    \item \textbf{Proveedor externo:} Servicio de fotografía profesional y edición
\end{itemize}

---

\section{Plan de Acciones del Proyecto}

\begin{table}[h!]
\centering
\renewcommand{\arraystretch}{1.2}
\setlength{\tabcolsep}{6pt}
\begin{tabularx}{\textwidth}{p{3cm} X p{2.5cm} p{2.2cm}}
\hline
\textbf{Acción} & \textbf{Descripción} & \textbf{Responsable} & \textbf{Tiempo} \\
\hline
Planificación & Definición del concepto visual, fecha y locación & Dirección RR.II & Febrero \\
Coordinación & Confirmación de directores y proveedor & Subdirección RR.II & Febrero \\
Logística & Preparación del espacio y lineamientos de vestimenta & Coordinación RR.II & Febrero \\
Ejecución & Sesión fotográfica institucional & Dirección / Proveedor & 28 feb. o 7 mar. \\
Postproducción & Edición y entrega del material fotográfico & Proveedor & 1 semana \\
Difusión & Uso y publicación del material institucional & RR.II & Marzo \\
\hline
\end{tabularx}
\caption{Plan de acciones del proyecto}
\end{table}

---

\section{Lugar y Fecha Tentativa}
\begin{itemize}
    \item \textbf{Lugar:} Campus PUCP o locación institucional previamente acordada.
    \item \textbf{Fecha tentativa:}
    \begin{itemize}
        \item Opción 1: 28 de febrero
        \item Opción 2: 7 de marzo
    \end{itemize}
\end{itemize}

---

\section{Recursos del Proyecto}

\subsection{Recursos Humanos}
\begin{itemize}
    \item Equipo del Área de Eventos y Relaciones Institucionales
    \item Directores de Económica
    \item Fotógrafo profesional
\end{itemize}

\subsection{Recursos Materiales}
\begin{itemize}
    \item Equipos fotográficos profesionales
    \item Espacio físico para la sesión
    \item Coffee break básico
\end{itemize}

---

\section{Indicadores de Desempeño}
\begin{itemize}
    \item Número de directores participantes en la sesión.
    \item Número de fotografías finales entregadas.
    \item Uso del material en canales institucionales.
    \item Alcance e interacción en redes sociales posteriores a la publicación.
\end{itemize}

---

\section{Duración del Proyecto}
La duración total del proyecto será de aproximadamente tres semanas, considerando las etapas de planificación, ejecución, postproducción y difusión del material.

---

\section{Presupuesto del Proyecto}

\begin{table}[h!]
\centering
\renewcommand{\arraystretch}{1.2}
\begin{tabularx}{\textwidth}{X p{4cm}}
\hline
\textbf{Concepto} & \textbf{Monto estimado} \\
\hline
Servicio de fotografía profesional & S/.  \\
Coffee break & S/. no \\
\hline
\textbf{Total estimado} & \textbf{S/. } \\
\hline
\end{tabularx}
\caption{Presupuesto estimado del proyecto}
\end{table}

---

\section{Riesgos y Medidas de Mitigación}
\begin{itemize}
    \item \textbf{Inasistencia de directores:} Confirmación anticipada y definición de fecha alternativa.
    \item \textbf{Problemas técnicos:} Coordinación previa y prueba del equipo fotográfico.
\end{itemize}

---

\section{Conclusiones}
El proyecto de Foto Institucional de Directores constituye una acción estratégica de alto impacto institucional, alineada con los objetivos de imagen, comunicación y posicionamiento de Económica, contribuyendo a consolidar una identidad organizacional sólida y profesional.

\end{document}
